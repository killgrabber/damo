
\chapter{Validierung}

\section{Analyse der spannkraftinduzierten Deformation}\

[explizit Unterschiede FDM und additiv herausstellen mit genauen Angaben über die Deformationen]

Mithilfe dieser Funktion können unterschiedliche Bauteilgeometrien und
Herstellungsprozesse auf ihre Deformation hin verglichen werden.
Bei einem additiv gefertigten Metallteil tritt bei den 
gleichen Spannungsstufen eine deutlich kleinere, 
aber dennoch erkennbare Deformation auf. Zusätzlich zeigt sich, 
dass bei verschiedenen Herstellungsprozessen trotz gleicher 
Geometrie unterschiedliche Verformungen auftreten.
Die verschiedenen Verformungen und Auswirkungen werden später in diesem
Kapitel behandelt.

\begin{figure}[H]
    \centering
    \includegraphics[width=0.9\textwidth]{images/AM_sp0_sp2_defo_plot.png}
    \caption{Differenz von mehreren Spannungsstufen bei einem AM Metall Bauteil}
    \label{fig:deformation_data_am}
\end{figure}

\subsection{Versuchsaufbau}

In Abbildung \ref{fig:versuchsaufbau} ist der Versuchsaufbau zur Datenerfassung 
zu sehen. Alle wichtigen Bestandteile sind nummeriert. Es folgt eine kurze Benennung
aller vorhandenen und notwendigen Teile:\\
1: Schraubstock Backen\\
2: Demonstratorbauteil\\
3: Scannerhalterung\\
4: Scanner LLT 30x0-25\\
5: Verschiebungsmesser\\
6: Laserlinie (Lila)\\
7: Schraubstock mit Kraftmesser\\
X: x-Achse\\
Y: y-Achse\\
Z: z-Achse\\

Der Scanner ist an dem Werkzeugkopf einer CNC-Fräse befestigt und wird 
in Richtung der X und Y Achse verschoben. So kann von dem kompletten Bauteil eine 
Pointcloud aufgenommen werden.

\begin{figure}[H]
    \centering
    \includegraphics[width=0.6\textwidth]{images/versuchsaufbau_foto.png.JPG}
    \caption{Versuchsaufbau}
    \label{fig:versuchsaufbau}
\end{figure}

\section{Zusätzliche Messinstrumente}

Zusätzlich zu dem Scanner werden noch mit weiteren Messinstrumente Daten erfasst.
In Abbildung \ref{fig:versuchsaufbau} unter der Nummer 5 ist ein mechanischer 
Verschiebungsmesser zu sehen. Dieser misst die Verschiebung der Backen des 
Schraubstocks. Der Schraubstock misst zusätzlich mit viel Kraft die Backen 
aufeinander pressen.

Hierzu wird die piezoelektrische Kraftmesstechnik verwendet.
Bei Krafteinwirkung auf Piezokristalle (z. B. Quarz, Bariumtitanat, BaTiO3) 
werden im Kristallgitter negative gegen positive Gitterpunkte
verschoben, sodass an den Kristalloberflächen
Ladungsunterschiede Q als Funktion der Kraft F
gemessen werden.
Piezoelektrische Kraftaufnehmer sind mechanisch sehr steif, 
sie erfordern Ladungsverstärker
zur Messsignalverarbeitung und sind hauptsächlich zur Messung dynamischer Vorgänge
mit einer kleineren Frequenz als 1 Hz geeignet. \cite{Czichos.2020}. 
Diese Kraftmesstechnik ist 
für unseren Einsatzzweck gut geeignet da sie eine hohe Empfindlichkeit bietet 
und in vielfältigen Formen und Größen hergestellt werden kann. Zur Aufbereitung 
der Ladung, die der piezoelektrische Sensor, liefert wurde ein Ladungsverstärker 
eingesetzt.\cite{Schwartz.2006}

\section{Messergebnisse}

Es wurden fünf Bauteile mit verschiedenen Spannungsstufen gemessen. Für jede 
Spannungsstufe wurde die Kraft, die auf das Bauteil wirkte sowie die Verschiebung 
des Schraubstocks gemessen.
Jede Spannungsstufe wurde durch Stufenweises anziehen des Schraubstocks erreicht.
Die Spannkraftkurve eines einzelnen Einspannvorgangs ist in 
Abbildung \ref{fig:single} zu sehen. 
In der Spannungskurve ist ein elastischer Bereich für das 
Bauteil zu sehen, in dem sich die Spannkraft zurückbewegt, nachdem kein 
Anzugdrehmoment mehr anliegt. Aus diesem Grund kann nicht der maximale Wert der Spannkraft angenommen werden, 
sondern es muss ein Wert gewählt werden der nach dem maximalen Ausschlag liegt.
Dieser wurde über die erste Ableitung der Spannkraftkurve gefunden. Sobald der 
Absolutwert der Steigung unter 0.0009 N fällt, wir die Spannkraft und Auslenkung an 
diesem Punkt gewählt. 0.0009 N wurde empirisch ermittelt um bei allen Bauteilen einen 
angemessenen Wert zu liefern.
Die Spannkraft wurde an zwei Achsen aufgenommen und zu der Gesamtkraft aufsummiert.

\begin{figure}[H]
    \centering
    \includegraphics[width=0.99\textwidth]{images/spannkraftstufen_single.png}
    \caption{Kraft- und Verschiebung der Spannungsstufe fünf bei einem FDM Bauteil}
    \label{fig:single}
\end{figure}

Diese maximalen Werte für die Spannkraft und Auslenkung wurden für jedes Bauteil 
akkumuliert und sind in Abbildung \ref{fig:akkumulated} dargestellt. 
Die mit dem FDM-Prozess hergestellten Bauteile wurden jeweils in sechs Spannungsstufen
gemessen. Zwischen den Stufen wurde versucht, eine konstante Kraft auf das Bauteil 
auszuüben. Durch den manuellen Prozess des Anziehens des Schraubstocks war dies 
jedoch nicht immer möglich.
Die Metallbauteile unterscheiden sich durch ihren Aufbau. 
Alle basieren auf dem gleichen 3D-Modell, besitzen jedoch unterschiedliche 
Stützstrukturen. Im Bauteil AM0 ist die vollständige Stützstruktur vorhanden,
während in den Bauteilen AM1 und AM2 die Stützstruktur in unterschiedlicher 
Tiefe ausgebohrt wurde. Die Bauteile sind in Abbildung \ref{fig:am_parts} dargestellt.

Das Bauteil AM0 wurde nur mit zwei Spannungsstufen gemessen, 
da bereits bei der zweiten Stufe über 2500 N Kraft erforderlich war, 
um das Bauteil nur minimal in x-Richtung zu deformieren. Dies zeigt, 
dass die Stützstruktur einen erheblichen Einfluss auf die Verformbarkeit 
eines Bauteils hat. Beim Bauteil AM1 wurden 2500 N erst nach vier Spannungsstufen 
erreicht. Vergleicht man die Verformung in x-Richtung mit der Verformung von AM2,
 zeigt sich, dass das Bauteil ohne Stützstruktur bei ähnlicher Krafteinwirkung 
 etwa doppelt so weit in x-Richtung deformiert wurde.

 Die FDM-Bauteile wurden mit deutlich weniger Kraft eingespannt. 
 Hier wurde bei etwa 250 N gestoppt, dennoch ist die Verschiebung der 
 Teile deutlich größer als bei den AM-Bauteilen. Diese Werte wurden aufgenommen, 
 um die visuelle Deformationserkennung zu validieren.

\begin{figure}[H]
    \centering
    \includegraphics[width=0.99\textwidth]{images/spannkraftstufen_akkumuliert.png}
    \caption{Akkumulierte Kraft- und Verschiebung jedes Bauteils}
    \label{fig:akkumulated}
\end{figure}

\begin{figure}[H]
    \centering
    \begin{minipage}{.33\textwidth}
      \centering
      \includegraphics[width=0.9\linewidth]{images/AM0_crop.JPG}
      \caption*{(a)}
    \end{minipage}%
    \begin{minipage}{.33\textwidth}
      \centering
      \includegraphics[width=0.9\linewidth]{images/AM1_crop.JPG}
      \caption*{(b)}
    \end{minipage}
    \begin{minipage}{.33\textwidth}
        \centering
        \includegraphics[width=0.9\linewidth]{images/AM2_crop.JPG}
        \caption*{(c)}
      \end{minipage}
      \caption{(a): AF Metallbauteil mit voller Stützstruktur, Bezeichnung: AM0.
      (b): AF Bauteil mit der halben Stützstruktur ausgebohrt, Bezeichnung: AM1.
      (c): AF Bauteil ohne Stützstruktur, Bezeichnung: AM2}
      \label{fig:am_parts}
\end{figure}

\section{Ergebnisse der optischen Deformationsanalyse}

In Abbildung \ref{fig:am_defos} und Abbildung \ref{fig:fdm_defos} sind die erkannten,
vertikalen Deformation grafisch dargestellt. Jeweils von Spannungsstufe null bis 
Spannungsstufen Sechs. Bei dem FDM Bauteil fehlt die Spannungsstufen fünf, 
diese ist leider bei der händischen Dateiname Vergabe überschrieben worden und konnte 
deshalb nicht ausgewertet worden. Aus diesem Grund ist eine so große Lücke in der 
Abbildung \ref{fig:fdm_defos}.
Außerdem sind große Unterschiede in der absoluten Deformation zu sehen. Zum Beispiel 
die rote Kurve in Abbildung \ref{fig:am_defos} die den Unterschied der Spannungsstufen 
null und eins angibt. Diese Kurve sollte näher an null der y-Achse liegen. 
Dies liegt an Ungenauigkeiten in dem Stitching Verfahren. In den Graphen ist also 
auf die Steigung der Deformationskurve zu achten. In der Steigung erkannt man das 
sich die Bauteile in mittleren Bereich nach außen hin deformiert haben und in 
den Randbereichen sich nach innen und dann wieder nach außen deformieren.
Außerdem ist im Vergleich der beiden Graphen zu sehen das sich das FDM Bauteil deutlich 
mehr verformt hat. Hier beträgt die größte Deformation über 150 Pixel.
Bei dem Metallbauteil, das mit der zehnfachen Kraft eingespannt wurde (250 nm vs. 2500 nm)
sind es nur knapp 40 Pixel. Trotzdem ist zu sehen das sich die Bauteile, die auch die 
gleiche Geometrie teilen, auf die gleiche Weise verformt haben.

\begin{figure}[H]
  \centering
  \includegraphics[width=0.95\textwidth]{images/am2_all_defos.png}
  \caption{Sechs Deformationsstufen bei einem AM Bauteil von 0 bis 2500 nm}
  \label{fig:am_defos}
\end{figure}

\begin{figure}[H]
  \centering
  \includegraphics[width=0.95\textwidth]{images/fdm2_all_defos.png}
  \caption{Fünf Deformationsstufen bei einem FDM Bauteil von 0 bis 250 nm}
  \label{fig:fdm_defos}
\end{figure}

