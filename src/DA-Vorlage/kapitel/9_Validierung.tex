
\chapter{Validierung}

Im folgenden Kapitel wird die Deformationerkennung analysiert und überprüft.
Zur Überprüfung werden Materialeigenschaften und aufgenommene Messwerte 
mit der erkannten Deformation verglichen.

\section{Messergebnisse}

Es wurden fünf Bauteile mit verschiedenen Spannungsstufen gemessen. Für jede 
Spannungsstufe wurde die Kraft, die auf das Bauteil wirkte sowie die Verschiebung 
des Schraubstocks gemessen.
Jede Spannungsstufe wurde durch Stufenweises anziehen des Schraubstocks erreicht.
Die Spannkraftkurve eines einzelnen Einspannvorgangs ist in 
Abbildung \ref{fig:single} zu sehen. 
In der Spannungskurve ist ein elastischer Bereich für das 
Bauteil zu sehen, in dem sich die Spannkraft zurückbewegt, nachdem kein 
Anzugdrehmoment mehr anliegt. Aus diesem Grund kann nicht der maximale Wert der Spannkraft angenommen werden, 
sondern es muss ein Wert gewählt werden der nach dem maximalen Ausschlag liegt.
Dieser wurde über die erste Ableitung der Spannkraftkurve gefunden. Sobald der 
Absolutwert der Steigung unter 0.0009 N fällt, wir die Spannkraft und Auslenkung an 
diesem Punkt gewählt. 0.0009 N wurde empirisch ermittelt, um bei allen Bauteilen einen 
angemessenen Wert zu liefern.
Die Spannkraft wurde an zwei Achsen aufgenommen und zu der Gesamtkraft aufsummiert.

\begin{figure}[H]
    \centering
    \includegraphics[width=0.99\textwidth]{images/spannkraftstufen_single.png}
    \caption{Kraft- und Verschiebung der Spannungsstufe fünf bei einem FDM Bauteil}
    \label{fig:single}
\end{figure}

Diese maximalen Werte für die Spannkraft und Auslenkung wurden für jedes Bauteil 
akkumuliert und sind in Abbildung \ref{fig:akkumulated} dargestellt. 
Die mit dem FDM-Prozess hergestellten Bauteile wurden jeweils in sechs Spannungsstufen
gemessen. Zwischen den Stufen wurde versucht, eine konstante Kraft auf das Bauteil 
auszuüben. Durch den manuellen Prozess des Anziehens des Schraubstocks war dies 
jedoch nicht immer möglich.
Die Metallbauteile unterscheiden sich durch ihren Aufbau. 
Alle basieren auf dem gleichen 3D-Modell, besitzen jedoch unterschiedliche 
Stützstrukturen. Im Bauteil AM0 ist die vollständige Stützstruktur vorhanden,
während in den Bauteilen AM1 und AM2 die Stützstruktur in unterschiedlicher 
Tiefe ausgebohrt wurde. Die Bauteile sind in Abbildung \ref{fig:am_parts} dargestellt.

Das Bauteil AM0 wurde nur mit zwei Spannungsstufen gemessen, 
da bereits bei der zweiten Stufe über 2500 N Kraft erforderlich war, 
um das Bauteil nur minimal in x-Richtung zu deformieren. Dies zeigt, 
dass die Stützstruktur einen erheblichen Einfluss auf die Verformbarkeit 
eines Bauteils hat. Beim Bauteil AM1 wurden 2500 N erst nach vier Spannungsstufen 
erreicht. Vergleicht man die Verformung in x-Richtung mit der Verformung von AM2,
 zeigt sich, dass das Bauteil ohne Stützstruktur bei ähnlicher Krafteinwirkung 
 etwa doppelt so weit in x-Richtung deformiert wurde.

 Die FDM-Bauteile wurden mit deutlich weniger Kraft eingespannt. 
 Hier wurde bei etwa 250 N gestoppt, dennoch ist die Verschiebung der 
 Teile deutlich größer als bei den AM-Bauteilen. Diese Werte wurden aufgenommen, 
 um die visuelle Deformationserkennung zu validieren.

\begin{figure}[H]
    \centering
    \includegraphics[width=0.99\textwidth]{images/spannkraftstufen_akkumuliert.png}
    \caption{Akkumulierte Kraft und Verschiebung, mit der jedes Bauteil deformiert wurde.}
    \label{fig:akkumulated}
\end{figure}

\begin{figure}[H]
    \centering
    \begin{minipage}{.33\textwidth}
      \centering
      \includegraphics[width=0.9\linewidth]{images/AM0_crop.JPG}
      \caption*{(a)}
    \end{minipage}%
    \begin{minipage}{.33\textwidth}
      \centering
      \includegraphics[width=0.9\linewidth]{images/AM1_crop.JPG}
      \caption*{(b)}
    \end{minipage}
    \begin{minipage}{.33\textwidth}
        \centering
        \includegraphics[width=0.9\linewidth]{images/AM2_crop.JPG}
        \caption*{(c)}
      \end{minipage}
      \caption{(a): AF Metallbauteil mit voller Stützstruktur, Bezeichnung: AM0.
      (b): AF Bauteil mit der halben Stützstruktur ausgebohrt, Bezeichnung: AM1.
      (c): AF Bauteil ohne Stützstruktur, Bezeichnung: AM2}
      \label{fig:am_parts}
\end{figure}

\section{Ergebnisse der optischen Deformationsanalyse}

In Abbildung \ref{fig:am_defos} und Abbildung \ref{fig:fdm_defos} sind die erkannten,
vertikalen Deformation grafisch dargestellt. Jeweils von Spannungsstufe null bis 
Spannungsstufen Sechs. Bei dem FDM Bauteil fehlt die Spannungsstufen fünf, 
diese ist leider bei der händischen Dateiname Vergabe überschrieben worden und konnte 
deshalb nicht ausgewertet worden. Aus diesem Grund ist eine so große Lücke in der 
Abbildung \ref{fig:fdm_defos}.
Außerdem sind große Unterschiede in der absoluten Deformation zu sehen. Zum Beispiel 
die rote Kurve in Abbildung \ref{fig:am_defos} die den Unterschied der Spannungsstufen 
null und eins angibt. Diese Kurve sollte näher an null der y-Achse liegen. 
Dies liegt an Ungenauigkeiten in dem Stitching Verfahren. In den Graphen ist also 
auf die Steigung der Deformationskurve zu achten. In der Steigung erkannt man das 
sich die Bauteile in mittleren Bereich nach außen hin deformiert haben und in 
den Randbereichen sich nach innen und dann wieder nach außen deformieren.
Außerdem ist im Vergleich der beiden Graphen zu sehen das sich das FDM Bauteil deutlich 
mehr verformt hat. Hier beträgt die größte Deformation über 150 Pixel.
Bei dem Metallbauteil, das mit der zehnfachen Kraft eingespannt wurde (250 nm vs. 2500 nm)
sind es nur knapp 40 Pixel, wie in Abbildung~\ref{fig:deformation_data_am} zu sehen. Trotzdem ist zu sehen das sich die Bauteile, die auch die 
gleiche Geometrie teilen, auf die gleiche Weise verformt haben.

\begin{figure}[H]
  \centering
  \includegraphics[width=0.95\textwidth]{images/am2_all_defos.png}
  \caption{Sechs Deformationsstufen bei einem additiv gefertigten Metallbauteil ohne
  Stützstruktur von 0 bis 2500 nm}
  \label{fig:am_defos}
\end{figure}

\begin{figure}[H]
  \centering
  \includegraphics[width=0.95\textwidth]{images/fdm2_all_defos.png}
  \caption{Fünf Deformationsstufen bei einem FDM Bauteil von 0 bis 250 nm}
  \label{fig:fdm_defos}
\end{figure}

\begin{figure}[H]
    \centering
    \includegraphics[width=0.9\textwidth]{images/AM_sp0_sp2_defo_plot.png}
    \caption{Differenz von zwei Spannungsstufen bei einem additiv gefertigten Metallbauteil, 
    das zur Hälfte mit Stützstrukturen gefüllt ist (Abbildung~\ref{fig:am_parts (b)}). }
    \label{fig:deformation_data_am}
\end{figure}

\section{Beurteilung der Ergebnisse}

Grundsätzlich ermöglicht die Methode die Erkennung und den Vergleich von Deformationen 
eines Bauteils. Die gemessene Deformation eines Bauteils entspricht den erwarteten Werten, 
abhängig von Material und Geometrie des Bauteils. FDM-gedruckte Kunststoffteile zeigen 
eine deutlich stärkere Verformung im Vergleich zu Metallteilen.
Bei den Metallteilen zeigt sich, dass das Vorhandensein einer Stützstruktur die
Verformung des Bauteils signifikant reduziert. 
Die unterschiedlichen Deformationen sind in Abbildung \ref{fig:materials} dargestellt. 
Das in Abbildung \ref{fig:am_parts} (a) gezeigte Bauteil konnte nicht analysiert werden, 
da durch die Stützstruktur keine korrekte Transformation zur stitchen berechnet werden konnte.
Bei den FDM-Bauteilen ist festzustellen, dass sie sich deutlich stärker verformen 
als die Metallteile. Beide FDM-Bauteile zeigen eine ähnliche Verformungsausprägung, 
jedoch tritt die Deformation, trotz identischer Geometrie, an unterschiedlichen Stellen auf. 
Dies könnte auf Unterschiede im Druckprozess zurückzuführen sein. 
Diese Beobachtung zeigt einen weiteren Nutzen der Methodik: Sie ermöglicht
nicht nur die Bestimmung des Ausmaßes der Deformation, sondern auch die 
Identifikation von Schwachstellen innerhalb eines Bauteils, die zu einer erhöhten 
 
Deformation führen.

\begin{figure}[H]
  \centering
  \includegraphics[width=0.95\textwidth]{images/compare_materials.png}
  \caption{Vergleich von Materialien und Bauteilgeometrien}
  \label{fig:materials}
\end{figure}

Dennoch ist die Deformationserkennung noch nicht perfekt. 
Fehler im Stitching Prozess führen zu starken Änderungen in der Deformationserkennung. 
Wie in Abbildung \ref{fig:errors} zu sehen ist, kann sich die Breite des Bauteils 
zwischen zwei Spannungsstufen unterscheiden. Er ist erkennbar, dass der Rand der einen 
Kontur, repräsentiert durch die magenta gefärbte Linie, immer größer ist als 
der Rand der anderen Kontur, hier in blau dargestellt.
Dieser Versatz entsteht, weil beim Stitching eine Transformation verwendet wurde die sich 
um wenige Pixel unterscheidet. Dadurch wächst auch die erkannte Deformationen.
Aus diesem Grund existieren in den Abbildungen \ref{fig:am_defos} und \ref{fig:fdm_defos}
Linien die nicht dem erwarteten Verhalten entsprechen. 
In Abbildung \ref{fig:fdm_defos} zum Beispiel, sollte die Deformation zwischen den 
Spannungsstufen eins und zwei 
(in rot dargestellt) am kleinsten sein. Stattdessen liegt die Kurve bei -50 Pixeln im Graph.

\begin{figure}[H]
  \centering
  \includegraphics[width=0.95\textwidth]{images/contours_matching_36.png}
  \caption{Vergleich von Konturen aus nicht korrekt zusammengefügten Bildern.}
  \label{fig:errors}
\end{figure}

Zusätzlich kann, bei Bildern ohne klare Konturen, das Stitching nicht korrekt durchgeführt
werden. Dies ist bei den additiv gefertigten Metallbauteil mit Stützstruktur der Fall 
(Abbildung \ref{fig:am_parts} (a)). Das Bild, das aus dem Scan dieses Bauteils erstellt wurde, 
ist in Abbildung \ref{fig:errorimage} zu sehen. Es lassen sich keine eindeutigen Konturen 
erkennen, die für das Stitching genutzt werden könnten.

\begin{figure}[H]
  \centering
  \includegraphics[width=0.95\textwidth]{images/am0.png}
  \caption{Bild eines Scans von einem Metallbauteil mit Stützstruktur.}
  \label{fig:errorimage}
\end{figure}


