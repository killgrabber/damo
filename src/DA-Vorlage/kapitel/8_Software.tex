
\chapter{Anwendung und Algorithmus}

Ziel dieser Arbeit ist es nicht nur eine Methodik zur optische 
Deformationserkennung zu entwickeln, sondern diese Methodik auch in einer 
Anwendung einfach nutzbar zu machen. 
Diese Kapitel dokumentiert diese Anwendung und geht auf Herausforderungen 
in der Entwicklung ein.

\section{Anwendung}

Die Anwendung beinhaltet verschiedene Funktionen, alle Funktionen 
können separat benutzt werden. Dadurch müssen zeitintensive Vorgänge, wie das Stitching,
nicht wiederholt werden, sondern Zwischenergebnisse können abgespeichert und 
neu geladen werden.
Die Anwendung bietet Funktionen, um Resultate in dem entsprechenden Dateiformat zu 
speichern. Soweit möglich werden Dateinamen automatisch ermittelt, 
daher ist es zu empfehlen von Anfang an mit einem einheitlichen Namensschema bei
den 3D-Scandaten zu arbeiten. 
Das Schema \glqq Bauteilbeschreibung \textunderscore Spannungsstufe
\textunderscore Scannerdurchlauf.ply\grqq~
hat sich bewährt. Ein Beispiel für den zweite Scan eines FDM-Bauteil bei der
vierten Spannungsstufen wäre also \glqq FDM0\textunderscore SP4\textunderscore 2.ply\grqq.

\newpage
\section{Dokumentation}

In Abbildung \ref{fig:software_screenshot} ist die Oberfläche der Anwendung dargestellt.

\begin{figure}[H]
    \centering
    \includegraphics[width=0.9\textwidth]{images/software_screenshot2.png}
    \caption{Anwendungsoberfläche}
    \label{fig:software_screenshot}
\end{figure}

Im folgenden werden die verschiedenen Anzeigen und Buttons in ihrer Funktionsweise beschrieben.
Über die Buttons \glqq Select Pointcloud\grqq~können Scandaten zum Konvertieren in Bilder
ausgewählt werden. Der Text neben dem Bild zeigt den Namen der ausgewählten 
Datei an. Die zuerst aufgenommenen Scandaten sollte hier als erstes ausgewählt werden.
Der Button \glqq Start PC conversion\grqq~startet die Konvertierung. Neben dem Button 
befindet sich eine Fortschrittsanzeige.
Wenn \glqq Show Pointclouds\grqq~gesetzt ist, werden die Scandaten vor dem 
Konvertieren in einem separaten Fenster angezeigt. So kann überprüft werden, ob die 
korrekten Scandaten ausgewählt wurde.
Wenn der Prozess abgeschlossen ist, werden die resultieren Bilder als Vorschau in der 
Anwendung angezeigt und die Option zum Speichern der Bilder aktiviert.
Zusätzlich wird nach dem Konvertierungsprozess die Schaltfläche 
\glqq Stitch converted\grqq~freigeschaltet. Durch diese Option können die 
Bilder direkt zusammengefügt werden, ohne das die Bilder extra gespeichert und 
eingeladen werden müssen. Wenn schon existierende Bilder zusammengefügt werden sollen, 
können die Schaltflächen \glqq Select top image\grqq~und \glqq Select bot image\grqq~
genutzt werden, um das obere und untere Bild auszuwählen. Auch hier wird der Name der 
ausgewählten Datei angezeigt. Die Dateiauswahl erfolgt über das 
Windows-Kontextmenü, der zuletzt verwendete Ordner wird hierbei erhalten, sodass das 
zweite Bild schneller ausgewählt werden kann. Dies gilt auch für das Speichern von Daten.
Über den Button \glqq Start stitching\grqq~wird der Stitching-Prozess gestartet. 
Auch hier wird der Fortschritt und das Endresultat, sobald es vorliegt, angezeigt.

\section{Visualisierungen}

Die auf der rechten Seite der Software zu sehenden Schaltfläche sind, für die 
Visualisierung der Ergebnisse zu benutzten. 
Messdaten, wie die Werte der Kraftmesser im Schraubstock oder die Verschiebung der 
Schraubstockbacken, können 
mit dem Button \glqq import tdms data\grqq~importiert werden. Nach dem Importieren 
können mehrere Datensets, zum Beispiel von verschiedenen Spannkraftstufen, mithilfe 
des Buttons \glqq combine tdms data\grqq~in einem Graph zusammengefasst werden.
Die in Kapitel \ref{validation} gezeigten Graphen wurden mithilfe dieser Funktion erzeugt.

Damit auch die CAD-Datei des additiv gefertigten Bauteils verglichen werden kann, 
existiert der \glqq Convert stl\grqq~Button. Hier wird eine .stl-Datei zu einem Bild 
konvertiert und kann über \glqq save stl\grqq~gespeichert werden.

Mithilfe der rechts zu sehenden Schaltflächen können Bilder auf ihre Deformation hin 
verglichen werden. Die resultieren Deformationsdaten werden automatisch als Textdatei
in einem Ordner names \glqq deformation\textunderscore data\grqq~gespeichert. 
Dieser Ordner wird automatisch in dem Verzeichnis erzeugt, in dem die Anwendung ausgeführt wird. 
Die erstellten Textdateien können mithilfe des Buttons \glqq Plot data\grqq~ausgewählt 
werden, und werden automatisch als Graph angezeigt.
Die in Kapitel~\ref{defodata} gezeigten Graphen wurde mit dieser Funktion erstellt.

\section{Optimierungen}

Beim Konvertieren und stitchen werden alle Prozesse, die nicht voneinander abhängig sind,  
nebenläufig ausgeführt. Das reduziert die Laufzeit der Prozesse.
Die Anwendung ist in Python geschrieben, rechenintensive Prozesse wurden aber mithilfe 
der Bibliothek \glqq Numba\grqq~in optimierten Maschinencode compiliert~\cite{numba}.
Dadurch ist der Konvertierungs- und Stitching-Prozess deutlich schneller geworden. 
Durch die Optimierungen konnte der Konvertierungsprozess von 49 s auf 41 s verschnellert 
werden. Bei dem Stitchprozess war die Verbesserung noch deutlicher. Hier benötigt die 
Anwendung ca. 30 s um zwei Konturen zu vergleichen.
Ohne Optimierungen braucht der Vergleich derselben Konturen 
ungefähr 600 Minuten.
