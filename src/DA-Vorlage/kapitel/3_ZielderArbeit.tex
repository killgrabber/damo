

\chapter{Methodik}

Wie schon beschrieben müssen additiv gefertigte Bauteile nachbearbeitet werden
bevor sie eingesetzt werden können. 
Um eine korrekte Nachbearbeitung gewährleisten zu können muss das additiv 
gefertigte Bauteil fixiert werden. 
Dies kann vorgenommen werden, indem das Bauteil in einen Schraubstock eingespannt wird.
Durch eine Einspannung kann das Bauteil so deformiert werden, 
dass die vorgesehene Nutzung nicht mehr möglich ist. 

Für die Beurteilung, ob ein Bauteil noch eingesetzt werden kann, ist es nötig die 
Deformation die auf das Objekt gewirkt hat zu erkennen. Wenn das Bauteil in einen 
Schraubstock eingespannt wird, wirkt eine Spannkraft über die Backen des Schraubstock
auf das eingespannte Bauteil.
Diese Kraft induziert eine Deformation auf das Bauteil. 
Diese Deformation soll optisch in einem Verfahren erkannt, dargestellt und 
anschließend ausgewertet werden.

\section{Verfahren zur optische Spannkraftdeformationserkennung}

Um die Deformation des Bauteils erfassen zu können wird das 3D-Objekt benötigt, 
dass als Grundlage für die AF diente. Zusätzlich werden optische Daten des Bauteils 
im deformiertem Zustand benötigt. Mit diesen beiden Daten kann der Unterschied 
ermittelt und ausgegeben werden.
Um auch minimale Deformationen erkennen zu können müssen die Daten des 
eingespannten Bauteils hinreichend genau sein. Deswegen wird ein Laserscanner zur 
Datenerfassung eingesetzt.
Wie schon beschrieben ist der Messbereich eines Laserscanners begrenzt. Da das 
Verfahren nicht auf eine Bauteilgröße beschränkt sein soll, müssen mehrere Scans als 
Eingabe akzeptiert und damit umgegangen werden.

\section{Vorgehen}

Das Verfahren, um eine Deformation in einem eingespannten additiv gefertigten
Bauteil, zu erkennen umfasst folgenden Schritte:

\begin{itemize}
    \item \textbf{Digitalisierung des Bauteils:}\\
        In diesem Schritt werden Messfehler und Ausreißer
        in den Scannerdaten entfernt. Anschließend werden die dreidimensionalen 
        Eingangsdaten in eine zweidimensionale Ansicht umgewandelt.
        Nach diesem Schritt liegen für jedes Bauteil mehrere Datensätze als 
        zweidimensionale schwarz weiß Bilder vor.
    \item \textbf{Stitching Methodik}\\
        Diese Bilder müssen nun zu einem einzelnen Bild zusammengefügt werden.
        Hierfür werden Gemeinsamkeiten in den sich 
        teilweise überlappenden Bildern gesucht und eine Transformation ermittelt,
        die auf eines von zwei Bildern angewendet werden kann, um sie zusammenzufügen.
    \item \textbf{Deformationserkennung}\\
        Sobald ein Bild für ein eingespanntes Bauteil vorliegt, können verschiedene 
        Zustände des Bauteils verglichen werden. Deformationen werden erkannt, in dem 
        die Länge und Breite des Bauteils verglichen wird, zusätzlich wird der 
        Abstand zwischen den Rändern des Bauteils berechnet. Es können verschiedene
        Spannungszustände verglichen werden, ein Vergleich mit dem initialen 
        3D-Design ist auch möglich. Hier werden auch Fehler erkannt die im 
        Fertigungsprozess entstehen erkannt. Der Unterschied zwischen Bauteilzuständen
        wird auch visuell ausgegeben und als Bild gespeichert.
        
\end{itemize}
























