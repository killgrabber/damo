% 1_EinleitungMotivation_tex
\chapter{Einleitung}

In den letzten Jahren hat die additive Fertigung (AF), auch bekannt als 3D-Druck, 
zunehmend an Bedeutung in der Industrie gewonnen~\cite{JADHAV20222094}. 
Diese innovative Technologie 
ermöglicht die schichtweise Herstellung komplexer Bauteilgeometrien und bietet im 
Vergleich zu traditionellen Fertigungsverfahren einen höheren Grad an 
Gestaltungsfreiheit. Trotz vieler Vorteile stehen Hersteller vor 
Herausforderungen bezüglich der Oberflächenqualität und Maßhaltigkeit 
der gefertigten Werkstücke~\cite{SCHNECK201919}.
Herausforderungen können die Bauteilgeometrie, die möglichen Materialien oder 
den Herstellungsprozess betreffen.
Die Herausforderungen bei der Bauteilgeometrie umfassen die minimalen 
Wandstärken sowie das maximale Bauteilvolumen, das von den technischen 
Spezifikationen des verwendeten 3D-Druckers abhängt. Für die AF stehen 
verschiedene Materialien zur Verfügung, 
die sich jedoch von den in der konventionellen, spanenden Fertigung 
verwendeten Werkstoffen unterscheiden. In der AF werden hauptsächlich Legierungen 
auf Basis von Titan, Aluminium, Nickel oder Chrom eingesetzt. Diese Materialien
müssen zudem in einer für das Verfahren geeigneten Form vorliegen, 
meistens in Form von Pulver oder Draht. Der Prozess zur Herstellung von 
Pulver oder Draht aus diesen Materialien ist kostenintensiv und schränkt
die Auswahl der verwendbaren Materialien ein. Aufgrund der 
schichtbasierten Natur der AF sind Stützstrukturen 
notwendig, um Bauteile mit geometrischen Formen, die Überhänge aufweisen, 
erfolgreich produzieren zu können. \label{drawbacks_af}
~\cite{Vranic.2017}

Einige dieser Herausforderungen, können durch Nachbearbeitung des Bauteils 
gelöst werden. Dazu gehört das Entfernen der Stützstrukturen und die Verbesserung
der Oberflächenqualität.
Für den Nachbearbeitungsschritt muss das Bauteil in seiner Position und Lage im Bauraum
der Werkzeugmaschine fixiert werden. Die hierzu aufzubringenden Spannkräfte 
können die filigranen Bauteile elastisch, in Extremfallen auch plastisch, verformen, 
sodass eine maßhaltige spangebende Nachbearbeitung verhindert wird. 
Um den Spannprozess und dessen Auswirkungen auf das Bauteil hinsichtlich 
der erzielbaren geometrischen Genauigkeit optimieren zu
können, ist eine Quantifizierung der spannkraftinduzierten Deformation notwendig.~\cite{newMethod}

Im Rahmen dieser Bachelorarbeit wird deshalb eine Methodik zum Erkennen und 
analysieren einer Deformation entwickelt. Das Ziel dieser Methodik ist es, 
mithilfe von optischen Informationen zu additiv gefertigten Bauteilen eine 
Deformation zu erkennen, wie sie zum Beispiel auftreten kann, wenn ein Bauteil 
in einem Schraubstock fixiert wird.

Das Verfahren soll auf verschiedenen Bauteilgeometrien anwendbar sein, daher werden 
möglichst wenig Annahmen über die Bauteilgeometrie getroffen.
Im Verlauf dieser Arbeit werden erst die theoretischen Grundlagen, die für die 
entwickelte Methodik notwendig sind dargestellt. 
Anschließend wird das Vorgehen der Methodik vorgestellt und die einzelnen 
Schritte erläutert. Das Verfahren besteht aus mehreren Schritten die sich in 
Datenerfassung, Datenaufbereitung, Stitchting und Deformationserkennung 
einteilen lassen.

Nach der Vorstellung des Verfahrens wird die Funktion der Methodik anhand eines 
Demonstratorbauteils das in einem Schraubstock fixiert und mit mehrere Kraftstufen 
angezogen wurde, validiert. Hier wird die erkannte Deformation bewertet und 
verschiedene Materialien und Herstellungsverfahren verglichen.

Ziel dieser Arbeit ist es zusätzlich, die entwickelte Methodik in einer einfach
benutzbaren Anwendung um zusetzten. Die Funktionsweise dieser Anwendung wird 
nach der Validierung der Methodik dokumentiert. Außerdem werden verschiedene 
Optimierungen dargelegt, die das Verfahren zeit- und speichereffizienter machen
und die Genauigkeit der Ergebnisse verbessern.

Zum Abschluss dieser Arbeit wird ein Fazit gezogen, in dem die erzielten 
Ergebnisse eingehend diskutiert werden. Zudem wird ein Ausblick auf 
mögliche zukünftige Entwicklungen und weiterführende Forschungsansätze gegeben.



