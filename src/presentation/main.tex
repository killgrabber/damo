% !TEX root = src/ausarbeitung/main.tex
\documentclass[11pt]{article}

\usepackage{cite}
\usepackage{subfiles}

\usepackage{geometry}
\usepackage{layout}
\geometry{
    left=3cm,
    right=2cm,
    top=2cm,
    bottom=2cm,
    showframe=false
    %textwidth=8cm, 
    %marginpar=3cm
    }

\usepackage{helvet}
\renewcommand{\familydefault}{\sfdefault}

\usepackage[ngerman]{babel}

\usepackage[]{graphicx}

\title{Präsentationsgliederung}
\author{Niklas Thieme}
\date{\today}

\begin{document}

\begin{figure}[t]
    \includegraphics[width=5cm]{images/1280px-Technische_Universität_Dortmund_Logo.svg.png}\hfill
    \includegraphics[width=5cm]{images/vmLogo.png}
\end{figure}

%\centering
\begin{center}
    \section*{Präsentation Gliederung}
\end{center}
%\maketitle

\begin{itemize}
    \item Einleitung, Titelseite
    \item Additive Fertigung
    \begin{itemize}
        \item Bedeutung von additiven Fertigungsmethoden
        \item Limitierungen
        \item Nachbearbeitung von Bauteilen
        \item Deformation durch Einspannen
    \end{itemize}
    \item Bauteil Digitalisierung
    \begin{itemize}
        \item Scanner (Messbereichlimitierung)
        \item Demonstratorbauteil
        \item Stitching (Top down)
    \end{itemize} 
    \item Optische Spannkraftdeformationsanalyse
    \begin{itemize}
        \item Vergleich entspannt und gespannten Zustand
        \item Optischer Vergleich (Bilder anzeigen)
        \item Vergleichswerte
    \end{itemize}
    \item Automatisierung: Punktwolke rein, Vergleich raus
    \item Evtl. Vergleich verschiedener Herstellungsverfahren und Werkstoffe
    \item Zeitplan
    \item Quellen (auch unter Bildern)
\end{itemize}


\newpage
\bibliography{sections/quellen}
\bibliographystyle{alpha}

\end{document}