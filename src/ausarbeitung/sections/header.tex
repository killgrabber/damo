\documentclass[../main.tex]{subfiles}
\begin{document}

\begin{table}[h]
    \centering
    \setlength{\tabcolsep}{12pt}
    \begin{tabular}{ll}
    Name      & Niklas Thieme                                                                 \\
    \\
    Mat.-Nr.  & 210015                                                                              \\
    \\
    Anschrift & \begin{tabular}[c]{@{}l@{}}Alfred-Nobel-Str. 3\\ 44149 Dortmund\end{tabular}        \\
    \\
    Thema     & \begin{tabular}[c]{@{}l@{}}Entwicklung einer Methodik zur optischen                \\
                Spannkraftdeformationsanalyse von additiv \\gefertigten Bauteilen\end{tabular}      \\
    \\
    Betreuer  & \begin{tabular}[c]{@{}l@{}}Prof. Dr.-Ing. Petra Wiederkehr\\ Jan Liß\end{tabular}  
    \\
\end{tabular}
\end{table}

\section*{Ausgangssituation und Ziel der Arbeit}

\subsection*{Ausgangssituation}
    Additive Fertigung (AF) ist eine Fertigungsmethode, die es ermöglicht 
    physische Objekte auf Basis eines digitalen 3D Modells zu erstellen. 
    Die Objekte werden gefertigt, indem das Material, meistens Plastik oder Metall, 
    Schicht für Schicht aufgebaut wird.
    Diese Technologie ist weit verbreitet und hat viele Anwendungszwecke und 
    Vorteile gegenüber spanenden Fertiungsmethoden. Die Vorteile in dieser Technologie
    liegen vor allem in der Design-Flexibilität, der einfachen Anpassung der Bauteile 
    und in der Minimierung von Materialverschwendung \cite{MEHRPOUYA202129}.

    Das Verfahren wird aufgrund der oben genannten Vorteile insbesondere bei der Herstellung von
    Prototypen verwendet. Die Flexibilität und einfache Anpassung der Bauteile bringen Vorteile in der
    Produktentwicklung gegenüber spanenden Fertigungsmethoden. \cite{NYS2023}
    
    AF bringt aber auch Nachteile mit sich. Kumbhar und Mulay beschreiben in ~\cite{Kumbhar2018} unter 
    anderem schlechte Oberflächenqualität, Physikalische Limitierungen des Fertigungsprozess 
    und limitierte Materialoptionen.
    
    Um diese Nachteile auszugleichen muss ein Bauteil häufig nachbearbeitet werden. 
    So können zum Beispiel die Oberflächen aufgebessert werden oder Materialreste entfernt werden.
    Um ein Bauteil zuverlässig nachbearbeiten zu können, kann es nötig sein es zu fixieren.
    Diese Fixierung kann sich auf die Struktur des Bauteils auswirken.

    \subsection*{Ziel der Arbeit}
    Ziel dieser Arbeit ist es, die Auswirkung einer Fixierung eines Bauteils zu analysieren.
    Speziell wird die Spannkraftdeformation analysiert, die auf ein Bauteil wirkt wenn es in einem
    Schraubstock eingespannt wird.
    Durch die Analyse sollen Materialen, Fertigungsmethoden und Designentscheidungen verglichen werden
    k"onnen. Betrachtet wird die Verformung des Bauteils die entsteht wenn ein Bauteil
    in ein Schraubstock eingespannt wird. Die Verformung wird gemessen indem das Bauteil vor und nach
    dem Einspannen gescannt wird. Aus Unterschieden in diesen beiden Scans kann die Deformierung bestimmt 
    werden. 

    \subsection*{Arbeitsschritte}
    Damit die Auswirkung auf das Bauteil genau bestimmt werden kann, ist eine genaue Analyse des Bauteils
    n"otig. Hierf"ur wird ein 3D Laserscanner der LLT30xx Baureihe von Micro-Epsilon benutzt. Mit einer 
    Maximalen Einzelpunktabweichung von $\pm$0.10 \% ist der Scanner ausreichend genau um die Bauteile
    vergleichen zu k"onnen \cite{SCANNER}.
    Der Scanner hat jedoch nur einen begrenzten Messbereich, um auch große Bauteile analysieren 
    zu k"onnen ist es notwendig mehrere Scans durchzuf"uren und diese
    zu einem digitalen Objekt zusammenzuführen.

    

\end{document}   
