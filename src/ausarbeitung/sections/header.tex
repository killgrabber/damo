\documentclass[../main.tex]{subfiles}
\begin{document}

\begin{table}[h]
    %\centering
    \setlength{\tabcolsep}{12pt}
    \begin{tabular}{ll}
    Name      & Niklas Thieme                                                                 \\
    \\
    Mat.-Nr.  & 210015                                                                              \\
    \\
    Anschrift & \begin{tabular}[c]{@{}l@{}}Alfred-Nobel-Str. 3\\ 44149 Dortmund\end{tabular}        \\
    \\
    Thema     & \begin{tabular}[c]{@{}l@{}}Entwicklung einer Methodik zur optischen                \\
                Spannkraftdeformationsanalyse von additiv \\gefertigten Bauteilen\end{tabular}      \\
    \\
    Betreuer  & \begin{tabular}[c]{@{}l@{}}Prof. Dr.-Ing. Petra Wiederkehr\\ Jan Liß,  M.Sc.\end{tabular}  
    \\
\end{tabular}
\end{table}

\section*{Problemstellung und Lösungsansatz}

%\cite{SCANNER}

Die additive Fertigung gewinnt in der Industrie zunehmend an Bedeutung, da sie im Vergleich zu
konventionellen Fertigungsverfahren einen höheren Grad an Gestaltungsfreiheit bietet. Je nach
eingesetztem Verfahren, ermöglicht der schichtweise Materialauftrag eine endkonturnahe
Herstellung komplexer Bauteilgeometrien wie belastungsoptimierter Leichtbaustrukturen
\cite{Schmidt.2017} Allerdings sind die Oberflächenqualität und die Maßhaltigkeit additiv gefertigter
Werkstücke limitiert, so dass eine Nachbearbeitung beispielsweise durch eine sich anschließende
Fräsbearbeitung notwendig sein kann \cite{Kumar.2023}.
\newline

Für den Nachbearbeitungsschritt muss das Bauteil in seiner Position und Lage im Bauraum der
Werkzeugmaschine fixiert werden. Die hierzu aufzubringenden Spannkräfte können die filigranen
Bauteile elastisch –in Extremfällen auch plastisch– verformen, sodass eine maßhaltige
spangebende Nachbearbeitung verhindert wird \cite{newMethod}. Um den Spannprozess und dessen
Auswirkungen auf das Bauteil hinsichtlich der erzielbaren geometrischen Genauigkeit optimieren
zu können, ist eine Quantifizierung der spannkraftinduzierten Deformation notwendig.
\newline

Im Rahmen dieser Bachelorarbeit soll deshalb eine Methodik zur optischen
Spannkraftdeformationsanalyse entwickelt werden. Auf Basis von geometrischen Ist-Daten eines
additiv gefertigten Bauteils die mit Hilfe eines Laserscanners (Micro-Epsilon, LLT3000-25/BL
\cite{MESSTECHNIK_2020}) aufgezeichnet werden (vgl. siehe \cite{Potthoff.2023}), ist eine Berechnung der
spannkraftinduzierten Verzerrung des Bauteils zu realisieren, indem die Daten eines entspannten
und gespannten Bauteils automatisiert miteinander verglichen werden sollen.
\newline

Da der Messbereich des einzusetzenden Lasers bei größeren Bauteilen einen limitierenden Faktor
darstellt, ist anfänglich eine geeignete Aufbereitung der Messdaten notwendig. Um eine
vollflächigen Darstellung der Oberfläche/Kontur des zu digitalisierenden Objekts zu erhalten, ist
deshalb eine Stitching-Methodik zu implementiert, die einzelnen Digitalisierungen ohne
Informationsverlust zusammenfügt (vgl. siehe \cite{SusanaBrandao.2014}, \cite{Sun.2021}). Das Benchmarking soll hierzu an einem
additiv gefertigten Demonstratorbauteil aus Edelstahl durchgeführt werden. Final soll die
entwickelte Methodik an unterschiedlichen Bauteilgeometrien validiert werden. Denkbar ist an
dieser Stelle auch eine Funktionsprüfung der Methodik anhand eines Vergleichs von Bauteilen aus
unterschiedlicher Herstellungsverfahren (respektive Werkstoffe) wie dem Laser Powder Bed
Fusion (Metall, Edelstahl) und dem Fused Deposition Modeling (Kunststoff, PLA).

\end{document}   
