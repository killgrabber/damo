\documentclass[../main.tex]{subfiles}
\begin{document}

\begin{table}[h]
    \centering
    \setlength{\tabcolsep}{12pt}
    \begin{tabular}{ll}
    Name      & Niklas Thieme                                                                 \\
    \\
    Mat.-Nr.  & 210015                                                                              \\
    \\
    Anschrift & \begin{tabular}[c]{@{}l@{}}Alfred-Nobel-Str. 3\\ 44149 Dortmund\end{tabular}        \\
    \\
    Thema     & \begin{tabular}[c]{@{}l@{}}Entwicklung einer Methodik zur optischen                \\
                Spannkraftdeformationsanalyse von additiv \\gefertigten Bauteilen\end{tabular}      \\
    \\
    Betreuer  & \begin{tabular}[c]{@{}l@{}}Prof. Dr.-Ing. Petra Wiederkehr\\ Jan Liß\end{tabular}  
    \\
\end{tabular}
\end{table}

\subsection*{Problemstellung und Lösungsansatz}

\begin{flushleft}
    Additiv gefertigte Bauteile sind aus der Industrie nicht mehr wegzudenken. 
    Sie erleichtern die Entwicklung mithilfe von Anschauungs.- und Funktionsprototypen, 
    eignen sich für Endprodukte sowie Werkzeuge und Formen.
    Damit additiv gefertigte Bauteile langfristig eingesetzt werden können, 
    bedarf es Daten, die beschreiben, wie sich die Bauteile durch die Nutzung verändern.
    Daten zur Verformung sind außerdem auch hilfreich,
    um Materialen oder Designentscheidungen vergleichen zu können ~\cite{Nobody06}.

    Test
\end{flushleft}
\end{document}   
