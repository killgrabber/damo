\documentclass[../main.tex]{subfiles}
\begin{document}

\begin{table}[h]
    \centering
    \setlength{\tabcolsep}{12pt}
    \begin{tabular}{ll}
    Name      & Niklas Thieme                                                                 \\
    \\
    Mat.-Nr.  & 210015                                                                              \\
    \\
    Anschrift & \begin{tabular}[c]{@{}l@{}}Alfred-Nobel-Str. 3\\ 44149 Dortmund\end{tabular}        \\
    \\
    Thema     & \begin{tabular}[c]{@{}l@{}}Entwicklung einer Methodik zur optischen                \\
                Spannkraftdeformationsanalyse von additiv \\gefertigten Bauteilen\end{tabular}      \\
    \\
    Betreuer  & \begin{tabular}[c]{@{}l@{}}Prof. Dr.-Ing. Petra Wiederkehr\\ Jan Liß\end{tabular}  
    \\
\end{tabular}
\end{table}

\subsection*{Ausgangssituation und Ziel der Arbeit}

\begin{flushleft}
    \textbf{Ausgangssituation}\\
    Additive Fertigung (im Text auch als 'AF' bezeichnet) ist eine Fertigungsmethode die es ermoeglicht 
    physische Objekte auf Basis eines digitalen 3D Modells zu erstellen. 
    Dafuer wird Material, meistens Plastik oder Metall, 
    Schicht fuer Schicht aufgetragen und so ein Objekt hergestellt.
    Diese Technologie ist weit verbreitet und hat viele Anwendungszwecke und 
    Vorteile gegenueber traditionellen Methoden. ~\cite{MEHRPOUYA202129}
    Die Vorteile liegen vor allem in der Design-Flexibiliteat, einfachen Anpassung der Bauteile 
    und in der minimierung von Materialverschwendung. ~\cite{MEHRPOUYA202129}
    \\
    Seit den ersten Additiven Fertigungsprozessen in den Achtizgerjahren
    hat sich die Technologie weiter entwickelt und zeigt immer noch ein Marktwachstum.
    Schneck beschreibt ausserdem in ~\cite{SCHNECK201919} umfangreich die Literatur zu Vorteilen von Additiven 
    Fertigungsmethoden. Er beschreibt mehrere Faktoren die einen erfolgreichen Einsatz von AF bedeuten.
    Darunter 'part perfomance' und 'manufacturing' als haeufig genutzt.
    \\
    AF ist ausserdem bei der herstellung von Protypen oft verwendet da die schon beschriebene Flexibiliteat 
    und einfache Anpassung des Modells es ermoeglicht Testmodelle vor dem finalen Endprodukt herzustellen.
    AF bringt aber auch Nachteile mit sich. In ~\cite{Kumbhar2018} beschrieben sind: 
    Schlechte Oberflaechenqualiteat, Physikalische Limitierungen des Bauteils und besonders Material
    das fuer den AF noetig ist.
    \\
    Um diese Nachteile auszugleichen muss ein Bauteil haeufig nachbearbeitet werden. Dafuer kann es noetig
    sein das Bauteil zu fixieren. Diese Fixierung kann sich auf das Bauteil auswirken.\\
    \newblock
    \textbf{Ziel der Arbeit}\\
    Ziel dieser Arbeit ist die Auswirkung einer Fixierung eines Bauteils zu analysieren.
    Speziell wird die Spannkraftdeformation analysiert die auf ein Bauteil wirkt wenn es in einen
    Schraubstock eingespannt wird.



    
    


    
    

    

    
\end{flushleft}
\end{document}   
