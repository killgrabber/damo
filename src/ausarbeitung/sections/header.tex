\documentclass[../main.tex]{subfiles}
\begin{document}

\begin{table}[h]
    \centering
    \setlength{\tabcolsep}{12pt}
    \begin{tabular}{ll}
    Name      & Niklas Thieme                                                                 \\
    \\
    Mat.-Nr.  & 210015                                                                              \\
    \\
    Anschrift & \begin{tabular}[c]{@{}l@{}}Alfred-Nobel-Str. 3\\ 44149 Dortmund\end{tabular}        \\
    \\
    Thema     & \begin{tabular}[c]{@{}l@{}}Entwicklung einer Methodik zur optischen                \\
                Spannkraftdeformationsanalyse von additiv \\gefertigten Bauteilen\end{tabular}      \\
    \\
    Betreuer  & \begin{tabular}[c]{@{}l@{}}Prof. Dr.-Ing. Petra Wiederkehr\\ Jan Liß\end{tabular}  
    \\
\end{tabular}
\end{table}

\section*{Ausgangssituation und Ziel der Arbeit}

\subsection*{Ausgangssituation}
    Additive Fertigung ('AF') ist eine Fertigungsmethode die es ermöglicht 
    physische Objekte auf Basis eines digitalen 3D Modells zu erstellen. 
    Die Objekte entstehen indem das Material, meistens Plastik oder Metall, 
    schicht für schicht aufgebaut wird.
    Diese Technologie ist weit verbreitet und hat viele Anwendungszwecke und 
    Vorteile gegenüber spanenden Fertiungsmethoden. Die Vorteile in dieser Technologie
    liegen vor allem in der Design-Flexibilität, der einfachen Anpassung der Bauteile 
    und in der Minimierung von Materialverschwendung ~\cite{MEHRPOUYA202129}.
    
    
    Seit den ersten Additiven Fertigungsprozessen in den Achtizgerjahren
    hat sich die Technologie weiter entwickelt und zeigt immer noch ein Marktwachstum.
    Schneck beschreibt außerdem in ~\cite{SCHNECK201919} umfangreich die Literatur zu Vorteilen von Additiven 
    Fertigungsmethoden. Er beschreibt mehrere Faktoren die einen erfolgreichen Einsatz von AF bedeuten, 
    darunter sind die Faktoren 'part perfomance' und 'manufacturing' am häufigsten für den erfolgreichen Einsatz von AF relevant.
    
    Additive Fertigung wird außerdem bei der Herstellung von Protypen oft verwendet, 
    da die schon beschriebene Flexibilität 
    und einfache Anpassung des Modells es ermöglicht Testmodelle herzustellen und 
    so das Produkt zu schnell zu verfeinern.    
    AF bringt aber auch Nachteile mit sich. Kumbhar und Mulay beschreiben in ~\cite{Kumbhar2018} unter anderem 
    Schlechte Oberflächenqualiteat, Physikalische Limitierungen des Fertigungsprozess und limitierte Materialoptionen.
    
    Um diese Nachteile auszugleichen muss ein Bauteil häufig nachbearbeitet werden. So können zum Beispiel die Oberflächen 
    aufgebessert werden oder Materialreste entfernt werden. Um ein Bauteil zuverlässig nachbearbeiten zu können, 
    kann es nötig sein es zu fixieren.
    Diese Fixierung kann sich auf die Struktur des Bauteils auswirken.

    \subsection*{Ziel der Arbeit}
    Ziel dieser Arbeit ist es, die Auswirkung einer Fixierung eines Bauteils zu analysieren.
    Speziell wird die Spannkraftdeformation analysiert die auf ein Bauteil wirkt wenn es in einem
    Schraubstock eingespannt wird.\\
    
Mein TODO: (nicht teil des Exposes)
\begin{itemize}
    \item Scanner beschreiben (Warum wird ein Scanner verwendet)
    \item Arbeitsplan (in Wochen)
\end{itemize}

\end{document}   
