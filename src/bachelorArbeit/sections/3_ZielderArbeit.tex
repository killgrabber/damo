\documentclass[../main.tex]{subfiles}
\begin{document}

\section{Ziel der Arbeit}

Wie schon beschrieben müssen additiv gefertigte Bauteile nachbearbeitet werden
bevor sie eingesetzt werden können. 
Um eine korrekte Nachbearbeitung gewährleisten zu können muss das additiv 
gefertigte Bauteil fixiert werden. Dies kann durch ein Einspannen in einem 
Schraubstock vorgenommen werden. 

\subsection{Einspannproblematik}

Durch das Einspannen kann das Bauteil so deformiert werden, dass die vorgesehene Nutzung nicht mehr möglich ist. 
Je nach verwendetem Werkstoff und Geometrie kann die Deformation unterschiedlich
ausfallen. 

\subsection{Erfassung der spannkraftinduzierten Deformation}

Für die Beurteilung, ob ein Bauteil noch eingesetzt werden kann, ist es nötig die 
Deformation die auf das Objekt gewirkt hat zu erkennen. Wenn das Bauteil in einen 
Schraubstock eingespannt wird, wirkt eine Spannkraft über die Backen des Schraubstock
auf das eingespannte Bauteil. 
Diese Kraft induziert eine Deformation auf das Bauteil. Diese Deformation soll 
optisch in einem Verfahren erkannt und dargestellt werden.

\subsection{Verfahren zur optische Spannkraftdeformationserkennung}

Um die Deformation des Bauteils erfassen zu können wird das 3D-Objekt benötigt, 
dass als Grundlage für die AF diente. Zusätzlich werden optische Daten des Bauteils 
im deformiertem Zustand benötigt. Mit diesen beiden Daten kann der Unterschied 
ermittelt und ausgegeben werden.
Um auch minimale Deformationen erkennen zu können müssen die Daten des 
eingespannten Bauteils hinreichend genau sein. Deswegen wird ein Laserscanner zur 
Datenerfassung eingesetzt.
Wie schon beschrieben ist der Messbereich eines Laserscanners begrenzt. Da das 
Verfahren nicht auf eine Bauteilgröße beschränkt sein soll, müssen mehrere Scans als 
Eingabe akzeptiert und damit umgegangen werden.

\end{document}