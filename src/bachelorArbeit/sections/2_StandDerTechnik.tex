\documentclass[../main.tex]{subfiles}
\begin{document}

\section{Stand der Technik}

\subsection{Additive Fertigung}

Additive Fertigung, unabhängig von dem Werkstoff, ist ein Bereich im dem viel
geforscht und innoviert wird. In fast jedem Industriebereich wird versucht ein
bestehendes Design oder Modell zu optimieren und verbessern. \cite{newMethod}
Sei es hinsichtlich Qualität oder Kosteneffizienz. Addtive Fertigung bietet bei 
dieser Optimierung viele Vorteile gegenüber spanenden Fertigungsverfahren, da 
Additive Fertigung einen höheren Grad der Gestaltungsfreiheit bietet. 
Additive Fertigung ist eine Resource die Benutzern ermöglicht komplexe 
Bauteilgeometrien zu erstellen ohne die Limitierung von konventionellen spanenden 
Herstellungsverfahren, wie hoher Materialverschleiß oder die Notwendigkeit von 
spezialisierten Werkzeugen. \cite{Vafadar.2021} 

Außerdem können mit Additiver Fertigung Stückzahlen drastisch reduziert werden.
Werkstücke können bei Bedarf gefertigt werden was die Notwendigkeit für Lagerstätten
größtenteils eliminiert. Zusätzlich können die Teile genau dort herstellt werden wo 
sie benötigt werden, was Lieferketten und Wartezeiten verkürzt.

Verschiedene Werkstoffe können mit Additiver Fertigung (AF) benutzt werden, darunter
sind Polymeren, Keramik und Metalle. Metalle haben vor allem in den letzten Jahren 
an relevanz gewonnen. Zusätzlich zu den schon genannten Vorteile von AF, 
bietet Metall als Werkstoff noch mehr Nutzen in der Industrie. Gegenüber Kunstoffen
produziert Metall weniger Abfall und kann eine höhere Qualität gewährleisten.
Zusätzlich dazu kommen die offensichtliche Vorteile von Metall gegenüber Polymeren: 
Höhere Hitzebeständigkeit und weniger anfällig für Verformungen.
 




\subsection{Laser Scanning}

\subsection{Reverse Enginnering}

\end{document}