\documentclass[../main.tex]{subfiles}
\begin{document}

\section{Validierung}

\subsection{Versuchsaufbau}

\begin{wrapfigure}{r}{0.6\textwidth}
    \includegraphics[width=0.6\textwidth]{images/versuchsaufbau_foto.png.JPG}
    \caption{Versuchsaufbau}
    \label{fig:versuchsaufbau}
\end{wrapfigure}

In Abbildung \ref{fig:versuchsaufbau} ist der Versuchsaufbau zur Datenerfassung 
zu sehen. Alle wichtigen Bestandteile sind nummeriert. Es folgt eine kurze Benennung
aller vorhandenen und notwendigen Teile:\\
1: Schraubstock Backen\\
2: Demonstratorbauteil\\
3: Scannerhalterung\\
4: Scanner LLT 30x0-25\\
5: Verschiebungsmesser\\
6: Laserlinie (Lila)\\
7: Schraubstock mit Kraftmesser\\
X: x-Achse\\
Y: y-Achse\\
Z: z-Achse\\

Der Scanner ist an dem Werkzeugkopf einer CNC-Fräse befestigt und wird 
in Richtung der X und Y Achse verschoben. So kann von dem kompletten Bauteil eine 
Pointcloud aufgenommen werden.

\end{document}